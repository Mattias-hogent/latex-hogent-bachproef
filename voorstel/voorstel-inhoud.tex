%---------- Inleiding ---------------------------------------------------------

\section{Introductie}%
\label{sec:introductie}
SmartEye is een netwerk serviceprovider die Infrastructure as a Service voor projecten vanaf 8 aansluitingen aanbiedt in hetzelfde gebouw, ongeacht de mix van individuele appartementen, studentenkamers, commerciële ruimten tot gemeenschap zoals kantoren, co-working, lift, tuin, parking.~\autocite{Smarteye2021}

Als telecomoperator dient SmartEye te voldoen aan een reeks verplichtingen, opgelegd door het BIPT (Belgisch Instituut voor postdiensten en telecommunicatie). Zo is op een niet-exhaustieve lijst onder ‘Verplichtingen rond (persoons)\-gegevens en privacy / Identificatie van eindgebruikers’ volgende verplichting terug te vinden: “Een operator is verplicht de abonnees op zijn elektronische communicatiediensten te identificeren (directe identificatiemethode), of er ten minste voor te zorgen dat de autoriteiten hen kunnen identificeren (indirecte identificatiemethode).”~\autocite{BIPT2023}

Stel nu dat bevoegde overheidsinstanties een threat kunnen backtracen tot een door SmartEye beheerde netwerkomgeving, kan SmartEye vervolgens de locatie (een ruimte binnen het gebouw) van de potentële threat actor identificeren?

Aansluitend bij deze vraag wil SmartEye logischerwijze ook proactief kunnen inspelen op potentiële vragen van de bevoegde overheidsinstanties door mogelijke threats ten allen tijde reeds zelf te onderkennen.

Aan de hand van een literatuurstudie en een proof of concept (PoC) willen we een antwoord geven op bovenstaande onderzoeksvraag. Met betrekking tot de bijkomende vraag van Smart\-Eye, zal de PoC zich echter wel beperken tot het detecteren van netwerkscans als threats, om vervolgens de threat actors te identificeren binnen de door SmartEye beheerde netwerken.
%---------- Stand van zaken ---------------------------------------------------

\section{Stand van zaken}%
\label{sec:state-of-the-art}

\subsection{Netwerkoverwegingen}

\paragraph{Publieke IPv4 adressen}
\emph{Kan SmartEye niet iedereen gewoon een publiek IPv4 geven? }

Neen. Zo goed als alle beschikbare IPv4-ranges werden immers reeds door het IANA~\autocite{ICANN2012} en de onderliggende Regional Internet Registries\\\autocite{Gerich1993} (RIR’s) uitgedeeld. Voor België gebeurde dit door het RIPE NCC.

Op 25 november 2019 echter, kondigde het \textcite{RIPE_NCC2019}  aan dat alle beschikbare ranges (/8, /22 of /22 als combinatie van /23’s en/of /24’s) uitgeput waren. Local Internet Registries (LIR’s) kunnen zich enkel nog voor een wachtlijst aanmelden om eventueel ooit één /24 te bekomen.

Getuige van dit schrijnend tekort is het statement van~\textcite{Neuckens2022} bij Proximus, dat zij hun laatste IPv4-blok van het RIPE NCC ontvingen in 2012.

\paragraph{NAT}
\emph{Hoe kunnen we dit oplossen binnen een IPv4-context?}

Traditionele Network Address Translation (NAT) stelt hosts binnen een privé-netwerk~\autocite{Moskowitz1996} in staat om transparant toegang te krijgen tot hosts in het externe netwerk (zoals het internet).

NAPT~\autocite{Holdrege1999} breidt het begrip ‘translation’ een stap verder uit door ook de transport identifier te vertalen (bijv. TCP en UDP poortnummers en ICMP query identifiers). Hierdoor kunnen de transport identifiers van een aantal privé-hosts gemultiplext worden tot de\\transport identifiers van een enkel extern adres. Met NAPT kan een verzameling hosts dus een enkel extern adres delen.

\paragraph{Carrier-Grade NAT}
SmartEye krijgt door een Internet Service Provider (ISP) per locatie één binnenkomend publiek IP, dat gedeeld wordt door meerdere tenants (huurders, eindgebruikers).\\ Daarom dienen alle NAT-translaties deterministisch~\autocite{Donley2014} te zijn ofwel gelogd~\autocite{Perreault2013} te worden indien deze dynamisch zijn. Zodoende wordt Carrier-Grade NAT geïmplementeerd. Identificatie van de potentiële threat actor is anders immers niet mogelijk.

\paragraph{Publieke IPv6 adressen}
\emph{Kunnen we dit oplossen door over te stappen op IPv6-context?}

Ja. Echter, vanuit een IPv6-context moet men nog steeds het IPv4 gebaseerde deel van het internet kunnen bereiken. Hoewel hiervoor door de Internet Engineering Task Force (IETF) oplossingen~\autocite{Arkko2011} werden uitgewerkt, blijft de vereiste geldig dat een en ander deterministisch dient te gebeuren of gelogd moet worden. Bovendien is het essentieel om aan te stippen dat SmartEye vandaag de dag IPv4 only is.

\subsection{Datacaptatie en -export}
\emph{Welke technologieën kunnen nu het loggen van NAT-translaties faciliteren en een zicht geven op verdachte netwerktrafiek?}

\paragraph{Netflow}
NetFlow werd door Cisco ontwikkeld als feature voor zijn routers. Essentiële versies zijn versie 5, die IPv4 only is, en versie 9, die zowel met IPv4 als IPv6 overweg kan. Waar bij versie 5~\autocite{Cisco2007} de velden qua inhoud vastliggen, is versie 9~\autocite{Claise2004} template gebaseerd waardoor de inhoud van de velden toewijsbaar is.
\paragraph{IPFIX}
IPFIX (Internet Protocol Flow Information Export)~\autocite{Aitken2013} is dan weer een door het IETF uitgewerkte standaard, gebaseerd op Netflow versie 9. De specificaties zijn terug te vinden in de RFC’s 7011 tot en met RFC 7015, en in RFC 5103.
\paragraph{sFlow}
Waar bij NetFlow en IPFIX een pakket-\\aggregatie op layer 3 plaatsgrijpt, vindt er bij sFlow sampling plaats op layer 2. sFlow~\autocite{Phaal2004} werd ontwikkeld door InMon Corporation.

% Voor literatuurverwijzingen zijn er twee belangrijke commando's:
% \autocite{KEY} => (Auteur, jaartal) Gebruik dit als de naam van de auteur
%   geen onderdeel is van de zin.
% \textcite{KEY} => Auteur (jaartal)  Gebruik dit als de auteursnaam wel een
%   functie heeft in de zin (bv. ``Uit onderzoek door Doll & Hill (1954) bleek
%   ...'')S

%---------- Methodologie ------------------------------------------------------
\section{Methodologie}%
\label{sec:methodologie}

We opteren voor een \textbf{proof of concept} (PoC) die zich zal beperken tot het detecteren van netwerkscans als threats, om vervolgens de threat actors te identificeren binnen de door SmartEye beheerde netwerken. Niet de volledigheid van de oplossing, maar de correctheid en werkbaarheid moet immers aangetoond worden.

Vanuit ook de businessvereiste om tot een oplossing te komen met een minimale kost, zal er voor geopteerd worden om de oplossing op basis van deterministische NAT-translaties in eerste instantie verder uit te werken. Aan het loggen van alle translaties hangt immers een stevig storageprijskaartje. Het BIPT vereist immers om over de historiek van een jaar te beschikken.

Op basis van een \textbf{literatuurstudie} zal eerst de haalbaarheid van een op deterministische NAT-translaties gebaseerde oplossing bekeken worden. Vervolgens zal de haalbaarheid effectief getest worden aan de hand van een PoC. Ook zal de literatuurstudie de mogelijkheden van IPv4 versus IPv6 onderzoeken.

\subsection{MoSCoW-analyse}

Een eerste MoSCoW-analyse brengt reeds volgende inzichten op functioneel vlak:
\begin{itemize}
    \item Must have
        \begin{itemize}
            \item Detecteren van threats
            \item Identificeren van threat actors
        \end{itemize}
    \item Should have
        \begin{itemize}
            \item DHCP-logging omdat louter de NAT-\\informatie onvoldoende is voor identificatie binnen een DHCP-gebaseerd netwerk.
        \end{itemize}
    \item Could have
        \begin{itemize}
            \item Deze werden in dit stadium van de opdracht nog niet gedetecteerd.
        \end{itemize}
    \item Won't have
        \begin{itemize}
            \item  Complexe threatdetecties. Enkel het onderkennen van netwerkscans maakt deel uit van deze PoC.
            \item Geolocatiebepalingen
        \end{itemize}
\end{itemize}

\subsection{Fasering}

\begin{itemize}
    \item Literatuurstudie
        \begin{itemize}
            \item Tijdbesteding: 50\%
            \item Deliverables: een duidelijk zicht geven op de mogelijke oplossingsrichtingen
        \end{itemize}
    \item Requirements-analyse
        \begin{itemize}
            \item Tijdbesteding: 10\%
            \item Deliverables: duidelijk aangeven wat  de preferente oplossing is
        \end{itemize}
    \item PoC
        \begin{itemize}
            \item Tijdbesteding: 40\%
            \item Deliverables: bewijs leveren van het feit dat de preferente oplossing implementeerbaar is en voldoet aan de gestelde requirements
        \end{itemize}
\end{itemize}

%---------- Verwachte resultaten ----------------------------------------------
\section{Verwacht resultaat}%
\label{sec:verwachte_resultaten}

Wanneer bevoegde overheidsinstanties een threat backtracen tot een door SmartEye beheerde netwerkomgeving, moet SmartEye vervolgens de locatie (een ruimte binnen het gebouw) van de potentële threat actor kunnen identificeren. Deze bachelorproef voorziet in deze wettelijke vereiste door een bewezen correcte en werkbare oplossing (PoC) aan te reiken die voldoet aan de vereisten die, én door de regulator gesteld worden, én door SmartEye. De meerwaarde van deze bachelorproef is bijgevolg aangetoond.

Deze bachelorproef zal mij veel bijkomend en diepgaand inzicht geven in het netwerkdomein. Het is immers duidelijk dat meerdere RFC’s, toch wel dé referentiedocumenten binnen het netwerkdomein, grondig zullen moeten doorgenomen en bestudeerd worden om tot een gefundeerde stellingname en keuze te komen.

