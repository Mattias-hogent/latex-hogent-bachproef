\chapter{Oplossingsanalyse}
\section{Processtappen}

\begin{figure}[H]
    \includesvg[width=\textwidth]{./graphics/2_Oplossing_b.svg}
    \caption[short description]{Vanuit de firewall service triggert nieuwe firewall data de procesketting van data collecteren en formatteren, analyseren en stockeren.}
    \label{fig:Oplossingsvoorstel}
\end{figure}

Voorafgaand aan de keuze en de configuratie van de applicatiesoftware is het nodig om een zicht te hebben op de verschillende processtappen die bij deze opzet nodig zijn.

\paragraph{}
De processtappen zijn, in chronologische volgorde, het collecteren en formatteren van beschikbare data, het analyseren van deze data en het stockeren ervan. Deze verschillende processtappen worden logischerwijze doorlopen telkens wanneer op firewall-niveau nieuwe flow data ter beschikking komt.

\section{Firewall service \& nieuwe firewall data}
Gezien SmartEye per locatie ervoor zorgt dat een binnenkomend publiek IP gedeeld wordt door meerdere tenants (huuders en/of eindgebruikers) en gezien de vereiste van de regulator om bij een potentiële threat actor zicht te hebben op de exacte locatie van deze tenant, blijkt uit de literatuurstudie dat Carrier-Grade NAT moet geïmplementeerd zijn. Dit betekent dat de firewall-omgeving Carrier-Grade NAT capable moet zijn.\\ Om dit te realiseren wordt gebruik gemaakt van de firewall API. Via Python scripting kunnen we enerzijds onze beschikbare poorten per IP in ranges opdelen. De grootte varieert hierbij in functie van het aantal tenants per gebouw. Anderzijds biedt Python scripting ons ook de mogelijkheid om via API calls onze opdeling in port ranges als rules in de firewall op te laden.

\paragraph{}
Gezien na het beschikbaar komen van nieuwe flow data deze moet kunnen gecollecteerd worden voor verdere afhandeling, moet de firewall-omgeving flow export capable gemaakt worden.\\ Dit kan gerealiseerd worden door configuration settings op de firewall aan te passen.

\paragraph{}
Deze beide functionaliteiten dienen aan de standaard firewall service toegevoegd te worden. In dit geval zal de firewall service geleverd worden door OPNsense.

\paragraph{}
Nieuwe firewall data komt ter beschikking telkens wanneer trafiek de firewall bereikt en zo als het ware de procesketting van data collecteren en formateren, data analyseren en data stockeren, triggert. Onze threat, een gelanceerde nmap scan, zal op deze manier dit ganse proces doorlopen.

\section{Data collecteren en formateren}
Voor het collecteren en het formateren van de uit de firewall geëxporteerde data doen we beroep op nProbe.

\section{Data analyseren}
Het analyseren van de data of threat detection gebeurt het behulp van de software ntopgn. Deze beschikt namelijk hiervoor over een ingebouwde functionaliteit. Enkel de gevoeligheid van de threat-detectie moet ingesteld worden door aan te geven waarover en wanneer (= bij het overschrijden van welke threshhold) een alert moet gegenereerd worden. Gegenereerde alerts worden vervolgens op een dashboard weergegeven. Normaal gesproken gaat ntopng zelf steeds een sessie opzetten richting nProbe. Gezien in een operationale uitrol van deze proof of concept nProbe in de verschillende gebouwen achter een firewall actief zal zijn, zal het nProbe zijn die een sessie richting ntopng moeten initiëren gezien de firewall de inkomende sessies blokkeert.

\section{Data stockeren}
Vanuit de businessvereiste om tot een oplossing te komen met een minimale kost, wordt geopteerd om een oplossing op basis van deterministische NAT-translaties uit te werken. Aan het loggen van alle translaties (als alternatief) hangt immers een stevig storage-prijskaartje. Het BIPT vereist immers om over de historiek van een jaar te beschikken.\\ Volledigheidshalve zien we als oplossing voor het stockeren van data heil in het gebruik van OpenSearch. Deze omgeving kan met behulp van een bulk export API vanuit ntopng gevoed worden.
