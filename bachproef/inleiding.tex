%%=============================================================================
%% Inleiding
%%=============================================================================

\chapter{\IfLanguageName{dutch}{Inleiding}{Introduction}}%
\label{ch:inleiding}

SmartEye is een netwerk serviceprovider die Infrastructure as a Service voor projecten vanaf 8 aansluitingen aanbiedt in hetzelfde gebouw, ongeacht de mix van individuele appartementen, studentenkamers, commerciële ruimten tot gemeenschap zoals kantoren, co-working, lift, tuin, parking.~\autocite{Smarteye2021}

\paragraph{}
Als telecomoperator dient SmartEye te voldoen aan een reeks verplichtingen, opgelegd door het BIPT (Belgisch Instituut voor postdiensten en telecommunicatie). Zo is op een niet-exhaustieve lijst onder ‘Verplichtingen rond (persoons)\-gegevens en privacy / Identificatie van eindgebruikers’ volgende verplichting terug te vinden: “Een operator is verplicht de abonnees op zijn elektronische communicatiediensten te identificeren (directe identificatiemethode),\\of er ten minste voor te zorgen dat de autoriteiten hen kunnen identificeren (indirecte identificatiemethode).”~\autocite{BIPT2023}

\paragraph{}
Stel nu dat bevoegde overheidsinstanties een threat kunnen backtracen tot een door SmartEye beheerde netwerkomgeving, hoe kan SmartEye vervolgens de locatie (een ruimte binnen het gebouw) van de potentële threat actor identificeren?

\paragraph{}
Aansluitend bij deze hoofdvraag wil SmartEye logischerwijze ook proactief kunnen inspelen op potentiële vragen van de bevoegde overheidsinstanties. Hoe kunnen mogelijke threats ten allen tijde reeds zelf onderkend worden?
% Het is gebruikelijk aan het einde van de inleiding een overzicht te
% geven van de opbouw van de rest van de tekst. Deze sectie bevat al een aanzet
% die je kan aanvullen/aanpassen in functie van je eigen tekst.


% TODO: Vul hier aan voor je eigen hoofstukken, één of twee zinnen per hoofdstuk

