\chapter{\IfLanguageName{dutch}{Stand van zaken}{State of the art}}%
\label{ch:stand-van-zaken}

% Tip: Begin elk hoofdstuk met een paragraaf inleiding die beschrijft hoe
% dit hoofdstuk past binnen het geheel van de bachelorproef. Geef in het
% bijzonder aan wat de link is met het vorige en volgende hoofdstuk.

% Pas na deze inleidende paragraaf komt de eerste sectiehoofding.

\section{Netwerkoverwegingen}

\subsection{Publieke IPv4 adressen}
\emph{Kan SmartEye niet iedereen gewoon een publiek IPv4 geven? }

Neen. Zo goed als alle beschikbare IPv4-ranges werden immers reeds door het IANA~\autocite{ICANN2012} en de onderliggende Regional Internet Registries\\\autocite{Gerich1993} (RIR’s) uitgedeeld. Voor België gebeurde dit door het RIPE NCC.

\paragraph{}
Op 25 november 2019 echter, kondigde het \textcite{RIPE_NCC2019}  aan dat alle beschikbare ranges (/8, /22 of /22 als combinatie van /23’s en/of /24’s) uitgeput waren. Local Internet Registries (LIR’s) kunnen zich enkel nog voor een wachtlijst aanmelden om eventueel ooit één /24 te bekomen.

\subsection{NAT44}
\emph{Hoe kunnen we dit oplossen binnen een IPv4-context?}

\begin{table}[!htbp]
    \caption{Private IPv4 adres ranges volgens \textcite{Moskowitz1996}}
    \label{tab:PrivateIPv4AdressRanges}
    \begin{tabular}{lllll}
        \hline
        \multicolumn{1}{|l|}{Eerste IP adres} & \multicolumn{1}{l|}{Laatste IP adres} & \multicolumn{1}{l|}{Aantal IP adressen} & \multicolumn{1}{l|}{Subnetmask} & \multicolumn{1}{l|}{CIDR} \\ \hline
        10.0.0.0                               & 10.255.255.255                       & 16777216                                    & 255.0.0.0                       & /8                        \\
        172.16.0.0                             & 172.31.255.255                       & 1048576                                     & 255.240.0.0                     & /12                       \\
        192.168.0.0                            & 192.168.255.255                      & 65536                                       & 255.255.0.0                     & /16
    \end{tabular}
\end{table}

Traditionele Network Address Translation (NAT) stelt hosts binnen een privé-netwerk~\autocite{Moskowitz1996} in staat om transparant toegang te krijgen tot hosts in het externe netwerk (zoals het internet).
Hierbij gebeurt een tijdelijke mapping van private op publieke IPv4-adressen. Een gevolg hiervan is dat het backtracen van een threat niet meer mogelijk is. Het IANA heeft voor de private IPv4-adressen dan ook specifieke IPv4 ranges gereserveerd zie \ref{tab:PrivateIPv4AdressRanges}

\subsection{NAPT}
Als uitbreiding op NAT kan met Network Address Port Translation (NAPT) een verzameling hosts een enkel extern adres delen. NAPT~\autocite{Holdrege1999} breidt immers het begrip ‘translation’ een stap verder uit door ook de transport identifier te vertalen (bijv. TCP en UDP poortnummers en ICMP query identifiers). Hierdoor kunnen de transport identifiers van een aantal privé-hosts gemultiplext worden tot de\\transport identifiers van een enkel extern adres.

\subsection{Carrier-Grade NAT}

\begin{table}[!htbp]
    \caption{Carier-Grade NAT adres range volgens \textcite{Weil2012}}
    \label{tab:Carrier-GradeNATAdressRanges}
    \begin{tabular}{lllll}
        \hline
        \multicolumn{1}{|l|}{Eerste IP adres} & \multicolumn{1}{l|}{Laatste IP adres} & \multicolumn{1}{l|}{Aantal IP adressen} & \multicolumn{1}{l|}{Subnetmask} & \multicolumn{1}{l|}{CIDR} \\ \hline
        100.64.0.0                            & 100.127.255.255                     & 4194304                                       & 255.192.0.0                   & /10
    \end{tabular}
\end{table}

SmartEye krijgt door een Internet Service Provider (ISP) per locatie één binnenkomend publiek IP, dat gedeeld wordt door meerdere tenants (huurders, eindgebruikers). Gezien uiteindelijk de identificatie van de potentiële threat actor vereist is, dient Carrier-Grade NAT geïmplementeerd te worden. Hierbij zijn alle NAT-translaties deterministisch~\autocite{Donley2014} of worden ze gelogd~\autocite{Perreault2013} indien ze dynamisch zijn. Het IANA heeft hier ook nog een specifieke IPv4 range voor gereserveerd zie \ref{tab:Carrier-GradeNATAdressRanges}

\subsection{Publieke IPv6 adressen}
We kunnen ook overstappen op IPv6. Echter, vanuit een IPv6-context moet men nog steeds het IPv4 gebaseerde deel van het internet kunnen bereiken. Hoewel hiervoor door de Internet Engineering Task Force (IETF) oplossingen~\autocite{Arkko2011} werden uitgewerkt, blijft de vereiste geldig dat een en ander deterministisch dient te gebeuren of gelogd moet worden. Bovendien is het essentieel om aan te stippen dat SmartEye vandaag de dag in het klantennetwerk IPv4 only is.

\subsection{Dual-Stack}
Om te zorgen dat men toch nog aan het IPv4 internet kunnen  we een native-IPv4 stack te samen met een native-IPv6 stack  draaien zoals in \textcite{Gilligan2005}. Dit lost echter de problemen die we met IPv4 hadden niet op en verdubbeld de configuratie en het onderhoud.

\subsection{Tunnelling}
\paragraph{6 in 4}
Hier gaan we een IPv6 packet encapsuleren in een IPv4 pakket zodat we dit over het bestaande IPv4 netwerk kunnen sturen zoals in \textcite{Gilligan2005}. Op deze manier moeten we niet dual-stack draaien en kunnen we snel IPv6 kan uitrollen over de bestaand IPv4 infrastructuur maar dit bereid niet voor op een uiteindelijke IPv6 transitie.

%TODO: 6RD

\paragraph{4 in 6}
In tegenstelling to 6 in 4 gaan  we hier een IPv4 packet encapsuleren in een IPv6 pakket op deze manier moeten opnieuw niet dual-stack draaien en in tegenstelling 6 in 4 bereid dit wel voor op een uiteindelijke IPv6 transitie en kunnen we native IPv6 verkeer gewoon doorsturen.

%TODO: DS-lite

\subsection{Translation}
\paragraph{6 to 4}
Hier gaan we een IPv6 packet vertalen naar een IPv4 pakket. op deze manier moeten opnieuw niet dual-stack draaien. Dit kan wel voor problemen zorgen en op een of andere manier moeten toestellen die geen IPv6 address heb toch bereikbaar zijn.
%TODO: NAT64 & DNS64

\paragraph{4 to 6}
Translatie van IPv4 naar IPv6 wordt niet echt gedaan alleen als tussenstap bij technieken zoals MAP-T en 464XLAT.

\section{Datacaptatie en -export}
\subsection{NetFlow}
NetFlow werd door Cisco ontwikkeld als feature voor zijn routers. Essentiële versies zijn versie 5, die IPv4 only is, en versie 9, die zowel met IPv4 als IPv6 overweg kan. Waar bij versie 5~\autocite{Cisco2007} de velden qua inhoud vastliggen, is versie 9~\autocite{Claise2004} template gebaseerd waardoor de inhoud van de velden toewijsbaar is.

\subsection{IPFIX}
IPFIX (Internet Protocol Flow Information Export)~\autocite{Aitken2013} is dan weer een door het IETF uitgewerkte standaard, gebaseerd op Netflow versie 9. De specificaties zijn terug te vinden in de RFC’s 7011 tot en met RFC 7015, en in RFC 5103.

\subsection{sFlow}
Waar bij NetFlow en IPFIX een pakket-\\aggregatie op layer 3 plaatsgrijpt, vindt er bij sFlow sampling plaats op layer 2. sFlow~\autocite{Phaal2004} werd ontwikkeld door InMon Corporation.
