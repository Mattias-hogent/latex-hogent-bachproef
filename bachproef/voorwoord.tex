%%=============================================================================
%% Voorwoord
%%=============================================================================

\chapter*{\IfLanguageName{dutch}{Woord vooraf}{Preface}}%
\label{ch:voorwoord}
\paragraph{}
Een bachelorproef is altijd een beetje reizen. Alles begint bij een idee dat je aanspreekt en krijgt vorm met de informatie die je vooraf inwint en de verhelderende gidsen die je onderweg ontmoet. Zo wordt een vaag idee concreet en werkelijk.
\paragraph{}
Het is dan ook een eer om langs deze weg mijn promotor, de heer Jeroen Courtens te bedanken. Hij was het die de bakens uitzette en mij begeleidde tot een behouden thuiskomst.
De meest verhelderende gids die ik onderweg mocht ontmoeten was zonder enige twijfel de heer Jonathan De Graeve, oprichter en CEO van SmartEye. Hij stelde niet enkel de infrastructuur van SmartEye ter beschikking, maar zat ook op geen enkel ogenblik verlegen om zijn brede en zeer diepgaande netwerkkennis met mij te delen.
\paragraph{}
Na een reis ben je altijd een ervaring rijker. Ook ik. Niet alleen op technisch vlak heb ik heel wat bijgeleerd, maar ook op redactioneel vlak. Een proof of concept eindigt immers niet na het enthousiast van de daken schreeuwen dat het werkt, maar eerder in de luwte van uren schrijfarbeid.
\paragraph{}
Ik wens u, de lezer, dan ook heel veel leesplezier toe. Moge het (goedaardige) netwerkvirus ook u even hard aansteken als het mij heeft aangestoken. Sinds de dag dat ik bij SmartEye mocht beginnen aan mijn stage en later bachelorproef, is naast kennis mijn enthousiasme over het boeiende IT-domein dat netwerken vormt, alleen maar dag na dag met rasse schreden toegenomen.
\paragraph{}
Mattias Verscheure
%% TODO:
%% Het voorwoord is het enige deel van de bachelorproef waar je vanuit je
%% eigen standpunt (``ik-vorm'') mag schrijven. Je kan hier bv. motiveren
%% waarom jij het onderwerp wil bespreken.
%% Vergeet ook niet te bedanken wie je geholpen/gesteund/... heeft
