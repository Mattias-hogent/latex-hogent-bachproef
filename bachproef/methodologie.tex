%%=============================================================================
%% Methodologie
%%=============================================================================

\chapter{\IfLanguageName{dutch}{Methodologie}{Methodology}}%
\label{ch:methodologie}

%% TODO: In dit hoofstuk geef je een korte toelichting over hoe je te werk bent
%% gegaan. Verdeel je onderzoek in grote fasen, en licht in elke fase toe wat
%% de doelstelling was, welke deliverables daar uit gekomen zijn, en welke
%% onderzoeksmethoden je daarbij toegepast hebt. Verantwoord waarom je
%% op deze manier te werk gegaan bent.
%%
%% Voorbeelden van zulke fasen zijn: literatuurstudie, opstellen van een
%% requirements-analyse, opstellen long-list (bij vergelijkende studie),
%% selectie van geschikte tools (bij vergelijkende studie, "short-list"),
%% opzetten testopstelling/PoC, uitvoeren testen en verzamelen
%% van resultaten, analyse van resultaten, ...
%%
%% !!!!! LET OP !!!!!
%%
%% Het is uitdrukkelijk NIET de bedoeling dat je het grootste deel van de corpus
%% van je bachelorproef in dit hoofstuk verwerkt! Dit hoofdstuk is eerder een
%% kort overzicht van je plan van aanpak.
%%
%% Maak voor elke fase (behalve het literatuuronderzoek) een NIEUW HOOFDSTUK aan
%% en geef het een gepaste titel.

\begin{figure}[!htbp]
    \includesvg[width=\textwidth]{./graphics/1- Analyse.svg}
    \caption[Requirements and Capability]{elaborate description}
    \label{fig:RequirementsCapability}
\end{figure}

\paragraph{}
Tijdens de \textbf{literatuurstudie} hebben we nagegaan hoe we het eenduidig verband tussen een threat actor op een netwerk en de eigenlijke tenant kunnen vastleggen. We hebben hiervoor twee pistes bewandeld.

\paragraph{}
Een eerste piste was om iedere tenant zijn eigen IP te geven. Hier liepen we tegen de limieten van het adresseringsbereik van IPv4. IPv6 met zijn astronomisch addresseringsbereik leek even de oplossing, maar hier botsten we op het feit dat IPv6 momenteel nog nauwelijks ingeburgerd is.

\paragraph{}
Een tweede piste was om de mogelijkheden van NAT’ing te bekijken. Hier vonden we uiteindelijk onder de vorm van Carrier-Grade NAT een oplossing voor ons probleem.

\paragraph{}
Om de werkbaarheid van Carrier-Grade NAT aan te tonen, opteerden we ervoor om een \textbf{Proof of Concept} te implementeren.

\paragraph{}
We kregen hierbij vanuit SmartEye enkele beperkende randvoorwaarden opgelegd. Als firewall dienden we gebruik te maken van OPNsense, verwerking van flow data diende te gebeuren met behulp van ntopng en alle software diende te draaien op het Ubuntu server operating system.

\paragraph{}
Om threats ten allen tijde te detecteren dienden we flow data te capteren op de firewall en te verwerken op ntopng. Deze laatste staat immers in voor de behavoural checks. Om echter de data op ntopng te krijgen, dienden we nProbe nog als tussenschakel toe te voegen die als data flowcollector fungeert. Eveneens diende bekeken te worden of en hoe de firewall in staat was om data flows te exporteren.

\paragraph{}
Naast het ten allen tijde detecteren van flow data, diende onze opzet ook nog in staat te zijn om threats eenduidig toewijsbaar te detecteren. Dit hebben we gerealiseerd door op de firewall Carrier-Grade NAT te implementeren. We realiseerden dit met behulp van een Python script dat via de firewall API port ranges per tenant onder de vorm van rules ging opladen.

\paragraph{}
Om over threats te kunnen rapporteren was het idee om flow data in OpenSearch op te slaan. Toen uit de literatuurstudie bleek dat het eenduidig verband tussen een threat actor op een netwerk en de eigenlijke tenant kon gerealiseerd worden met behulp van Carrier-Grade NAT, leek de noodzaak om flow data op te slaan in functie van rapportering te vervallen. Echter Carrier-Grade NAT bleek slechts sluitend als oplossing zolang het aantal tenants per gebouw beperkt bleef tot enkele tientallen in plaats van vele honderden.